\chapter{数据与主要方法}
\enchapter{Data and methods}

\section{数据}
\ensection{Datasets}

\subsection{船载ADCP数据}
\ensubsection{ADCP data}

本文使用了 Nuka Arctica 邮轮船载声学多普勒流速剖面仪
(Acoustic Doppler Current Profiler, ADCP)测量的上层海洋高空间分辨率流速资料。

……

\section{分析方法}
\ensection{Analysis methods}

\subsection{Wave-Vortex分解}
\ensubsection{Wave-Vortex}

本文使用了 Wave-Vortex 分解方法对船载 ADCP 测量结果进行了分析,
参照Bühler提出的波-涡分解方法\cite{buhler2014wave-vortex},采用一维动能谱的Wave-Vortex动能谱分解办法。
这种方法能够解决由空间尺度上波动和涡动的尺度重叠导致二者在空间谱上信号混杂的问题。
但分解方法和其他湍流分析一样有严格的假设:
(1)数据测量稳定均匀,(2)各向同性,(3)数据在观测空间尺度上具有周期性。

首先,我们将跨航迹流速和沿航迹动能谱通过Helmholtz分解,转化为旋转和辐散两部分,将其写成谱函数的形式:

\begin{equation}
  k\frac{\mathrm{d}\widehat{F}^{\phi}}{\mathrm{d}k} - \mathrm{d}\widehat{F}^{\psi}
    = -\frac{\widehat{C}^{v}}{2}
\end{equation}

\begin{equation}
  k\frac{\mathrm{d}\widehat{F}^{\psi}}{\mathrm{d}k} - \mathrm{d}\widehat{F}^{\phi}
    = -\frac{\widehat{C}^{u}}{2}
\end{equation}

式中,($\widehat{F}{\phi}$, $\widehat{F}{\psi}$)为计算过程谱函数,
($\widehat{C}^{u}$, $\widehat{C}^{v}$)分别为沿航迹函数和跨航迹函数,$k$为水平波数……
