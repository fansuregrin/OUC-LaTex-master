\chapter{结论与展望}
\enchapter{Conclusion and future work}

\section{结论}
\ensection{Conclusion}

北大西洋作为大西洋深层水的发源地,对全球气候变化有着重要影响[6]。本文围绕中尺度过程、亚中尺度过程和近惯性内波之间的能量交换过程展开了一系列研究。首先,利用船载ADCP对Iceland Basin 和Irminger Sea的中尺度-亚中尺度运动和海洋内波的动能波数谱进行研究。讨论了中尺度过程、亚中尺度过程之间的转换尺度影响因素,季节变化特征和深度依赖性。其次,本文结合平衡流波数谱和三阶结构函数深入分析了两个海盆的能量串级和能量注入问题。进一步,由于亚中尺度能量注入问题和海洋斜压不稳定密切相关,本文利用Irminger Sea 阵列观测周围的水下滑翔机数据分析了亚中尺度锋面的季节特征和变化规律,并进一步分析了海洋多种类型的不稳定。最后,利用Iceland Basin 和Irminger Sea锚定浮标观测序列分析了中尺度、亚中尺度和近惯性内波之间的能量交换特征。本文得到以下结论:

北大西洋中尺度-亚中尺度运动与海洋内波,存在明显的季节变化特征。中尺度—亚中尺度运动在冬季动能谱斜率更接近于-5/3,在其他季节接近-3,说明除冬季外,上层海洋主要由准地转湍流过程调控。海洋内波在冬季最强,在夏季最弱。二者之间的转换尺度在夏季最小为28千米,在冬季海洋内波在各个尺度上超过中尺度-亚中尺度运动。这种转换尺度的巨大变化主要由海洋内波过程引……  

\section{展望}
\ensection{Future work}

北大西洋作为大西洋深层水的发源地,对全球气候变化有着重要影响[6]。本文围绕中尺度过程、亚中尺度过程和近惯性内波之间的能量交换过程展开了一系列研究。首先,利用船载ADCP对Iceland Basin 和Irminger Sea的中尺度-亚中尺度运动和海洋内波的动能波数谱进行研究。讨论了中尺度过程、亚中尺度过程之间的转换尺度影响因素,季节变化特征和深度依赖性。其次,本文结合平衡流波数谱和三阶结构函数深入分析了两个海盆的能量串级和能量注入问题。进一步,由于亚中尺度能量注入问题和海洋斜压不稳定密切相关,本文利用Irminger Sea 阵列观测周围的水下滑翔机数据分析了亚中尺度锋面的季节特征和变化规律,并进一步分析了海洋多种类型的不稳定。最后,利用Iceland Basin 和Irminger Sea锚定浮标观测序列分析了中尺度、亚中尺度和近惯性内波之间的能量交换特征。本文得到以下结论:

北大西洋中尺度-亚中尺度运动与海洋内波,存在明显的季节变化特征。中尺度—亚中尺度运动在冬季动能谱斜率更接近于-5/3,在其他季节接近-3,说明除冬季外,上层海洋主要由准地转湍流过程调控。海洋内波在冬季最强,在夏季最弱。二者之间的转换尺度在夏季最小为28千米,在冬季海洋内波在各个尺度上超过中尺度-亚中尺度运动。这种转换尺度的巨大变化主要由海洋内波过程引……  