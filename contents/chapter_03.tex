\chapter{Iceland Basin和Irminger Sea动能波数谱的季节特征和能量串级}
\enchapter{Iceland Basin and Irminger Sea}

本文首先使用船载SADCP数据,对Iceland Basin和Irminger Sea两个海盆的动能谱特征进行了分析。
论文的主要研究对象是中尺度过程、亚中尺度过程和近惯性内波之间的相互作用,
而亚中尺度过程和中尺度过程的定义是依据地转关系在空间上对海洋物理过程的划分,
因此对他们的研究将从空间特征着手……

\section{动能谱的总体特征}
\ensection{Overall characteristics of the kinetic energy spectrum}

在具体分析动能谱的季节变化之前,本文对所有可用船载航迹的动能谱进行垂向平均后取他们的均值,
得到Iceland Basin和Irminger Sea具有代表性的动能谱。
Iceland Basin的动能谱在……

\section{动能量谱的季节变化}
\ensection{Seasonal variation of the kinetic energy spectrum}

\begin{table}[htb]
  \centering\small
  \tablecaption{船载ADCP仪器设置}{Setup of shipborne ADCP instruments}
  \label{tab:exampletable}
  \begin{tabular}{ccccc}
    \toprule[1.5pt]
    ~~~ 观测时期 ~~~   & ~~~ 空间分辨率 ~~~  & ~~~ 时间分辨率 ~~~  
      & ~~~ 观测深度 ~~~ & ~~~ 仪器频率 ~~~  \\
    (年) & (km) & (min) & (m) & (kHz)  \\
    \midrule[0.5pt]
    1998--2022 & 3 & 7 & 400 & 150 \\
    无线表 & 1.5 & 5 & 800 & 75   \\
    \bottomrule[1.5pt]
  \end{tabular}
\end{table}

